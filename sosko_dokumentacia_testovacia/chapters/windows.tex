\section{DHCP}
\paragraph{}
Inštaláciu sme vykonali vo Windows service manager - Add Roles and Features, vybrali si možnosť DHCP server.
\paragraph{}
Pre konfiguráciu sme klikli na TOOLS a následne DHCP.Zobrazilo sa nám okno s ponukou , my sme vybrali náš server, IPv4 a možnosť new scope. Spustil sa New ScopeWizard. V prvom kroku  sme napisali názov pravidla na prideľovanie IP adries. Ďalej sme zvolili  rozsah IP adries a masku.
\\
\\
Rozsah IP adries od 192.168.0.1 po 192.168.0.254\\
Maska 255.255.255.0
\paragraph{}
Následne sme využili možnosti pridať výnimku z predtým zadaného rozsahu, teda adresy ktoré sa nebudu prideľovať napriek tomu, že sú zo nami  zadaného rozsahu v predchádzajúcom kroku. Ide o adresy serverov 192.168.0.2 a 192.168.0.3.
\paragraph{}
Potom  sme zvolili aký dlhý čas si server bude pamätať IP adresy ktoré niekomu pridelil. Stačilo nám 5 hodín (dĺžka cvičenia aj s rezervou).
\paragraph{}
Nakoniec sme nastavili bránu na „192.168.1.1”, pridali sme IP adresy našich DNS serverov, teda  „192.168.1.2“ a „192.168.1.3”. , a dokončili inštaláciu kliknutím na Finish.

\section{DNS}
\paragraph{}
V prvom rade sme si zvolili Master a Slave. Master je server1 (192.168.0.2) a Slave server2 (192.168.0.1)
\paragraph{}
DNS master nainštalujeme pomocou Windows Server Manager. Klikneme na Manage , vyberieme možnosť Add roles and features daľej Role-based or feature-basedinstallation, zobrazí sa  zoznam serverov, my vyberieme náš server a zvolímezo zoznamu rolesDNS Server a dokončíme inštaláciu.
\paragraph{}
Po inštalácií DNS balíka sme sa dostali cez Tools -\textgreater{} DNS -\textgreater{} Configure a DNS server -\textgreater{} Create a forward lookupzone k vytvoreniuprimárnej forward lookupzóny sos1.local , nastavili sme aj  nech záznamy preposiela na Slave 192.168.0.3.
\paragraph{}
Prešli sme k inštalácii DNS Slave.Postup ako pri DNS Master avšak  DNS server bolo potrebné nastaviť na slavemode. Vybrali sme  Tools -\textgreater{} Forward lookupzones -\textgreater{} New zone. Hneď v prvom kroku sme vybrali  možnosť nie Primaryzone ale Secondaryzone a taktiež  meno zóny .V ďalšom kroku určíme DNS Masterserver,v našom riešení ma  IP 192.168.0.2 .Dokončíme vytváranie zóny pomocou Next a Finish. Onedlho si Slave stiahne záznamy z DNS Master servera.

\section{NTP}
\paragraph{}
Na spustenie NTP na Windows servery sme museli vykonať zmeny v registroch. Spustíme okno RUN  (WIN+R) , kde napíšeme regedit. Následne sa dostaneme cestou HKEY\_LOCAL\_MACHINE | SYSTEM | CurrentControlSet | Services | W32Time | TimeProviders | NtpServer	až k hodnote Enabled , ktorá bola nastavená na 0 , a my ju zmeníme na 1.Využijeme opäť win+R , zadáme w32tm /config /update,čím vlastne spustíme NTP server na danom zariadení.
\paragraph{}
Na aplikáciu  zmien sme reštartovali Windows Timeservice príkazom zadaným do commandline:
\\
\\
net stop w32time \&\& net start w32tim.

\section{NAT}
\paragraph{}
Nainštalovanie sme vykonali vo Windows Server Manageri, kde cez Add Roles and Features. Pridali sme
\\
\\
imidž
\paragraph{}
potvrdili sme službu Routing a následne sme ju nainštalovali. Pri inštalácii zvolíme sie\v{t}ový adaptér eth0, ktorý je pripojený k internetu. Po inštalácii je NAT plne funkčné, ale je potrebné prida\v{t} NAT záznamy na porte 53.  Ďalej sme ho potrebovali nakonfigurovať v control panel -\textgreater{} Administrative tools -\textgreater{} Routing and Remote Access. Po kliknutí na NAT, vyberieme záložku s adaptérom, ktorý je pripojený k internet. V Address Pool je potrebné nastavi\v{t} from, čo znamená našu počiatočnú public adresu 158.193.139.74 a to, čo je naša koncová adresa 158.193.139.75 a maska 255.255.255.0. V záložke services and ports je potrebné prida\v{t} 4 nové záznamy NAT pre DNS(Master-Slave, TCP-UDP).

\section{Web server}
\paragraph{}
Webserver ISS (Internet Information Server) sme pridali cez windows server manager tlačidlom Addroles and features, kde sme vyhľadali Web Server ISS a pokračujeme ďalej. Pri ponuke Role Services
\paragraph{}
Následne nainštalujeme služby na server. Po úspešnej inštalácii sa IIS objaví na \v{l}avom paneli v server manager-i. Klikneme na ikonu IIS a v zozname dostupných serverov sa zjaví jeden - ten, na ktorom uskuto\v{c}\v{n}ujeme konfiguráciu. Klikneme na\v{n} pravým tla\v{c}idlom myši a z ponuky zvolíme možnos\v{t} Internet Information Services (IIS) Manager. Otvorí sa nové okno, v ktorého \v{l}avom paneli sa nachádza náš server. Rozbalíme jeho ponuku a klikneme na Sites. Pravý klik na Default Web Site nám ponúkne viacero možností vrátane nastavenia webstránky a pridania novej.

\paragraph{}
Po inštalácii sa nachádza ISS v ľavom paneli vo Windows Server Manager-i. Po kliknutí na tools v pravom hornom rohu klikneme Internet Information Services (ISS) manager. A po rozkliknutí na ľavom rohu je už vytvorená default sites. Otvoriť ju je možné zadaním do browseru “localhost”.

\section{Poštový server}
\paragraph{}
V server manageri klikneme na tools a v záložne DNS, nasmerujume sa ku DNS severu a vytvoríme nové záznamy pre mail server. Cname záznam mail 158.193.139.74, dva MX(Mail exchanger) záznamy 0 mail sos3.local a 10 mail sos3.local.
\paragraph{}
Zo stránky mailenable.com stiahneme standart edition. Začneme inštaláciou stiahnutého balíčka, zaklikneme web mail service(server). V nasledujúcich krokoch napíšeme do Domain Name: sos3.local a DNS host: 192.168.0.2 a smtp port: 25. Počas inštalácie nám vybehne tabuľká, kde odklikneme aby sa mailserver inštaloval ako webserver ISS. V server manageri po kliknutí servers -> localhost -> system -> diagnose si skontrolujeme či všetky políčka sú pass, čo nám značí že mail enable funguje. V ďalšom kroku servers -> localhost -> services and connectors  a na SMTP klikneme pravým tlačidlom a klikneme na properties. V záložne general nastavíme default mail domain name čo je v našom prípade mail.sos3.local. Ďalej v záložke smart host nastavíme IP/DOMAIN 158.193.139.74. Po reštarte serveru vidíme že, všetky service sú running.
