\section{Inštalácia operačného systému Debian 8.6.0 x64}
\subsection{Inštalácia serverov}
\paragraph{}
netinst
Stiahnut image\\
vytvoriť virtuálku\\
nastaviť virtuálku\\
nainštalovať debian\\

Stiahli sme si obraz disku Debian 8.6.0 stable 64 bitovú verziu . Obraz disku (iso súbor) sme pripojili k diskovej mechanike VM a spustili sme inštaláciu OS. Hostname–každá VM má svoje meno odvodené od jej funkcie napr. všetky servery majú označenie „SX“ a desktopy „DX“, kde X je poradové číslo servera/desktopu. Domain – nastavili sme pridelenú doménu: sos3.local. Servery nemajú grafické rozhranie, iba textové. 

b. Študent prejde postupne jednotlivými nastaveniami, ktoré ponúka nástroj VB.
c. Pre jednotlivé VM sme správne nastavili sieťové adaptéry a virtualizačné parametre, napríklad PAE/NX  a paravirtualizačné rozhranie v SYSTEM ACCELERATION.
d. Každej VM sme nastavili správny počet a funkčnosť sieťových adaptérov: 2 pre firewall, 1 pre servery a desktopy
e. Zdieľaný adresár sme na VM nepridávali.

\subsection{Inštalácia desktopov}
\paragraph{}
rovnako ako servery, akurát s xfce

\section{Základná konfigurácia}
\paragraph{}
nejaké balíčky, čo sme inštalovali na všetky počítače

\section{Firewall}
\paragraph{}
ako sme robili firewall v iptables

\subsection{Spúšťanie iptables skriptu po štarte}
\paragraph{}
iptables skript po štarte

\section{VLAN}
\paragraph{}

\section{DNS}
\paragraph{}

\section{DHCP}
\paragraph{}

\section{NTP}
\paragraph{}

\section{Web server}
\paragraph{}

\subsection{Joomla}
\paragraph{}

\subsection{Mediawiki}
\paragraph{}

\section{Poštový server}
\paragraph{}

